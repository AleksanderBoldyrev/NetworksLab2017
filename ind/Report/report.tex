\include{settings}

\begin{document}	% начало документа

% Титульная страница
\include{titlepage}

% Содержание
%\include{ToC}


\section{Цель работы}
Ознакомиться с принципами программирования собственных протоколов, созданных на основе TCP и UDP.

\section{Краткое описание выполненных базовых работ по TCP и UDP}


В ходе выполнения лабораторных работ были написаны простейшие клиент-серверные приложения на базе протоколов TCP и UDP.
В приложениях TCP создается сокет, ставиться на прослушивание и при подключении клиента создается отдельный сокет, по которому клиент общается с сервером. 

Для инициализации, запуска и завершения TCP-сервера необходимо выполнить следующие системные вызовы:

\begin{enumerate}
\item socket() - создание сокета
\item bind() - привязка созданного сокета к заданным IP-адресам и портам
\item listen() – перевод сокета в состояние прослушивания
\item accept() - прием поступающих запросов на подключение и возврат сокета для нового соединения
\item recv() - чтение данных от клиента из сокета, полученного на предыдущем шаге
\item send() - отправка данных клиенту с помощью того же сокета
\item shutdown() - разрыв соединения с клиентом
\item close() - закрытие клиентского и слушающего сокетов
\end{enumerate}

TCP-клиенты выполняют следующую последовательность действий для открытия соединения, отправки и получения данных, и завершения:

\begin{enumerate}
\item socket() - создание сокета
\item connect() - установка соединения для сокета, который будет связан с серверным сокетом, порожденным вызовом accept()
\item send() - отправка данных серверу
\item recv() - прием данных от сервера
\item shutdown() - разрыв соединения с сервером
\item close() - закрытие сокета
\end{enumerate}

	Так же был реализован сервер, поддерживающий работу с несколькими клиентами.
Для этого при подключении клиента создается поток, который в котором создается сокет для общения с клиентом.
	
	В приложениях UDP сервер принимает сообщение от клиента и отправляет сообщение об успешной доставке. UDP протокол не подразумевает логических соединений, поэтому не создается слушающего сокета.

Реализация UDP-сервера имеет следующий вид:
\begin{enumerate}
\item socket() - создание сокета
\item bind() - привязка созданного сокета к заданным IP-адресам и портам
\item recvfrom() - получение данных от клиента, параметры которого заполняются функцией
\item sendto() - отправка данных с указанием параметров клиента, полученных на предыдущем шаге
\item close() - закрытие сокета
\end{enumerate}


UDP-клиент для обмена данными с UDP-сервером использует следующие функции:
\begin{enumerate}
\item socket() - создание сокета
\item recvfrom() - получение данных от сервера, параметры которого заполняются функцией
\item sendto() - отправка данных с указанием параметров сервера, полученных на предыдущем шаге
\item close() - закрывает сокет
\end{enumerate}


Проверено 2 способа написания клиента:
\begin{itemize}
\item С использованием функции connect
\item Без использования функции connect
\end{itemize}
В первом случае устанавливается соединение и клиент, и сервер работают аналогично TCP. Во втором случае при отсутствии доступа к серверу сообщение об ошибке не возникает, и клиент считает, что данные отправлены корректно.


\section{Индивидуальное задание}


Разработать приложение-клиент и приложение-сервер электронной почты. TCP-сервер реализован на Linux, TCP-клиент на Windows, UDP - наоборот. Реализация в зависимости от платформы будет различаться только в используемых библиотеках, так в Windows-реализации добавятся два системных вызова (WSAStartup(), WSACleanup()).

\textit{Основные возможности.}

Серверное приложение должно реализовывать следующие функции: 
\begin{enumerate}
\item Прослушивание определенного порта
\item Обработка запросов на подключение по этому порту от клиентов
\item Поддержка одновременной работы нескольких почтовых клиентов через механизм нитей
\item Приём почтового сообщения от одного клиента для другого
\item Хранение электронной почты для клиентов 
\item Посылка клиенту почтового сообщения по запросу с последующим удалением сообщения 
\item Посылка клиенту сведений о состоянии почтового ящика 
\item Обработка запрос на отключение клиента 
\item Принудительное отключение клиента
\end{enumerate}

Клиентское приложение должно реализовывать следующие функции:  
\begin{enumerate}
\item Установление соединения с сервером
\item Передача электронного письма на сервер для другого клиента
\item Проверка состояния своего почтового ящика
\item Получение конкретного письма с сервера
\item Разрыв соединения
\item Обработка ситуации отключения клиента сервером
\end{enumerate}

\textit{Настройки приложений.}
Разработанное клиентское приложение должно предоставлять пользователю настройку IP-адреса или доменного имени сервера электронной почты, номера порта, используемого сервером, идентификационной информации пользователя. Разработанное серверное приложение должно предоставлять пользователю настройку списка пользователей почтового сервера. 

\textit{Методика тестирования.}
Для тестирования приложений запускается сервер электронной почты и несколько клиентов. В процессе тестирования проверяются основные возможности приложений по передаче и приёму сообщений.

\section{Дополнительное задание}

Попробовать проанализировать код с помощью статического и динамического анализатора. Когда закончите индивидуальные задания, проверьте исходный код своих программ с помощью clang-tidy и cppcheck. При запуске утилит включайте все доступные проверки. После этого проверьте с помощью valgrind свои программы на предмет утечек памяти и неправильного использования многопоточности. Опишите все найденные ошибке в отчете, а также укажите, как их можно исправить. (в качестве средства анализа выбран PVS-studio)



\section{Разработанный прикладной протокол}

Базовое расположение реализации протокола – файл API.h, хранящий закодированные команды enum-типа, а также строковые расшифровки этих команд.

Все команды отправляются в формате <API[STATE]|numArg|\{args|\}|>
Все сообщения отправляются в формате $\hat{}$<id>$\hat{}$<from>$\hat{}$<date/time>$\hat{}$<len>$\hat{}$<state>$\hat{}$ , и могут в произвольном количестве передаваться в качестве аргументов. Как сообщения, так и команды сериализуются перед отправкой в строки и десериализуются после получения.
Команды: 
\begin{enumerate}
\item START -> после посылки данной команды сервер отправляет код SERV\_OK, подтверждающий успешное создание соединения с клиентом.
\item INIT -> данное состояние существует только для отображения меню клиента (не аутентифицированного).
\item EXIT -> после посылки данной команды сервер отправляет код SERV\_OK, подтверждающий успешный разрыв соединения с клиентом и закрытие сокета.
\item REG [uname, passw] -> после посылки данной команды и указанных аргументов сервер отправляет код SERV\_OK, подтверждающий успешное создание учетной записи пользователя.
\item LOG [uname, passw] -> после посылки данной команды и указанных аргументов сервер отправляет код SERV\_OK, подтверждающий успешный вход в заданную учетную запись.
\item LUG -> после посылки данной команды сервер отправляет код SERV\_OK, подтверждающий успешный выход клиента из учетной записи.
\item SND [uname, mes] -> после посылки данной команды и указанных аргументов сервер отправляет код SERV\_OK, подтверждающий успешное создание сообщения и отправки его указанному клиенту.
\item DEL\_US -> после посылки данной команды сервер отправляет код SERV\_OK, подтверждающий успешное удаление пользователя (удаление только учетной записи пользователя, находящегося в системе).
\item DEL\_MES [mesID] -> после посылки данной команды сервер отправляет код SERV\_OK, подтверждающий успешное удаление сообщения с данным ID.
\item SH\_UNR -> после посылки данной команды сервер отправляет код SERV\_OK, а также все непрочитанные сообщения из почтового ящика.
\item SH\_ALL -> после посылки данной команды сервер отправляет код SERV\_OK, а также все сообщения из почтового ящика.
\item SH\_EX [mesID] -> после посылки данной команды сервер отправляет код SERV\_OK, а также конкретное сообщение из почтового ящика по данному ID.
\item RSND [uname, mesID] -> после посылки данной команды сервер отправляет код SERV\_OK, подтверждающий успешную пересылку сообщения одним пользователем другому. 
\item INSYS -> данное состояние существует только для отображения меню клиента (аутентифицированного).
\end{enumerate}

Также во всех указанных случаях сервер имеет возможность послать ответ NO\_OPERATION, свидетельствующий об ошибке (например при некорректном выполнении операции на стороне сервера).

\begin{figure}[H]
	\begin{center}
		\includegraphics[scale=0.7]{format}
		\caption{Форматы пересылаемых пакетов для TCP и UDP реализаций} 
		\label{pic:format} % название для ссылок внутри кода
	\end{center}
\end{figure}
На данном рисунке представлен формат пакетов, передаваемых для TCP и UDP протоколов.
Стандартно, для TCP-варианта пакет содержит поле длины сообщения неизменной длины (10 символов, в UTF-8 кодировке - 10 байт), далее поле кода операции, количества аргументов и сами аргументы (их размер не фиксирован), а также разделители между нефиксированными полями.

Для UDP-варианта пакет содержит поле номера пакета (10 символов, в UTF-8 кодировке - 10 байт), поле длины сообщения неизменной длины (10 символов, в UTF-8 кодировке - 10 байт), далее поле кода операции, количества аргументов и сами аргументы (их размер не фиксирован), а также разделители между нефиксированными полями. В отличие от TCP, здесь пакеты делятся на передаваемые подпакеты равной длины, фиксируемой в протоколе (например, 1024 байта) - исходные пакеты делятся на подпакеты в соответствии с этим значением с учетом служебных полей номера пакета и длины. Поле длины присутствует только в первом передаваемом подпакете, в остальных - отсутствует. Если в самом последнем подпакете сообщения остается свободное пространство, во избежание приема мусора в конце данных этого подпакета вставляется спецсимвол окончания данных (код 002 ASCII).


\section{Тестирование приложения на основе TCP}

Для тестирования приложения запускался сервер и несколько клиентов. Проверялись все команды поддерживаемые сервером в различных комбинациях.
	В результате тестирования ошибок выявлено не было, из чего можно сделать вывод, что приложение работает корректно.

\lstinputlisting[
	label=code:code_1,
	caption={Пример вывода информации при работе с клиентом},% для печати символ '_' требует выходной символ '\'
]{code_1.txt}


\section{Тестирование приложения на основе UDP}

Для тестирования приложения запускался сервер и несколько клиентов. Проверялись все команды поддерживаемые сервером в различных комбинациях.
	В результате тестирования ошибок выявлено не было, из чего можно сделать вывод, что приложение работает корректно.

\begin{figure}[H]
	\begin{center}
		\includegraphics[scale=0.7]{wire}
		\caption{Пример наблюдения передаваемых пакетов в WireShark} 
		\label{pic:wire} % название для ссылок внутри кода
	\end{center}
\end{figure}

Как можно видеть на данном рисунке, по сети передаются пакеты по протоколу UDP, содержащие помимо служебной информации непосредственно передаваемые данные.

\begin{figure}[H]
	\begin{center}
		\includegraphics[scale=0.5]{profiler}
		\caption{Исследование узких мест приложения с помощью профайлера} 
		\label{pic:profiler} % название для ссылок внутри кода
	\end{center}
\end{figure}

С помощью профилировщика, встроенного в Visual Studio, было выявлено, что процессор в наибольшей степени обрабатывает запросы Sleep() и вывода информации на экран, после них - блокирующая функция recvfrom(). 
Каждый клиентский поток работает в бесконечном цикле, поэтому мы принуждаем его ожидать для того, чтобы другие потоки могли работать с разделяемыми системными ресурсами и ресурсами, разделяемыми между остальными потоками нашего приложения. Именно поэтому без вызова Sleep() будет слишком много вызовов блокирования мьютекса при том, что данные за этот промежуток времени не изменились, что приведет к потере производительности приложения.

\section{Дополнительное задание}

В качестве средства статического и динамического анализа был выбран плагин PVS-studio для Visual Studio, пример для приложения UDPServer.

\begin{figure}[H]
	\begin{center}
		\includegraphics[scale=0.7]{pvs}
		\caption{Пример выводимых возможных уязвимостей приложения в PVS-Studio} 
		\label{pic:pvs} % название для ссылок внутри кода
	\end{center}
\end{figure}

В приложении замечена одна критическая и семь некритических уязвимостей. Критическая - утечка памяти, в связи с тем, что память, выделенная для flags, не освобождается. Решение - освобождать память путем вызова delete[] flags.

\lstinputlisting[
	label=code:code_2,
	caption={Пример исправления ошибки, найденной статическим анализатором [flags]},% для печати символ '_' требует выходной символ '\'
]{code_2.txt}

Другой пример - переопределение типа USHORT, зарезервированного в системе - исправляется изменением имени типа.

 \lstinputlisting[
	label=code:code_3,
	caption={Пример исправления ошибки, найденной статическим анализатором [USHORT]},% для печати символ '_' требует выходной символ '\'
]{code_3.txt}

PVS-Studio - сигнатурный анализатор исходного кода, а соответственно имеющий базу признаков объектов исходного кода, на основе которой и происходит выявление ошибок и уязвимостей. Однако такого рода базу необходимо постоянно поддерживать и обновлять, так как с обновлением версий программ могут появляться новые типы уязвимостей.
PVS имеет находить логические ошибки (например ошибки копирования кода), в отличие от статических анализаторов (которые находят синтаксические ошибки). Минус этого анализатора - он может находить много лишних ошибок (семантические ошибки могут возникать в правильном коде).
PVS-Studio позволяет находить ошибки копирования кода, ошибки некорректного использования переменных и т.д.
PVS-Studio при этом не позволяет находить ошибки типа "неиспользуемый код", либо ошибки типа "ошибки синхронизации".

Для проекта TcpServer для Linux был использован Valgrind - динамический анализатор. Были исследованы утечки памяти (memcheck) на примере функции Deserialize().
Листинг результатов работы программы предтавлен в приложении.

Как можно видеть, в функции Deserialize() обнаружена утечка памяти - для переменной args не вызывается функция освобождения памяти. Решением проблемы является добавление стрки "delete[] args" после использования этой переменной.
 \lstinputlisting[
	label=code:code_5,
	caption={Пример исправления найденной утечки памяти},% для печати символ '_' требует выходной символ '\'
]{code_5.txt}

\section{Выводы}
В данной лабораторной работе было реализовано клиент-серверное приложение электронной почты. Данная система обеспечивает параллельную работу нескольких клиентов.

В случае данного варианта индивидуального задания, было необходимо организовать работу почтового клиент-серверного приложения, реализующего многопоточность для общения с множеством клиентов, а также реализовать собственный протоколы хранения и сетевого обмена. В ходе разработки вставал ряд вопросов о реализации собственного протокола, решения для которых находились по мере развития проекта. 

Например, вставал вопрос о том, каким образом хранить сообщения на сервере (решением являлось организация директории в корне приложения сервера, в котором хранятся файлы сообщений и учетной записи). Другой пример – протокол сетевого обмена. В связи с тем, что не все передаваемые данные имеют фиксированную длину, было решено при передаче передавать строки с разделителями между аргументами, таким образом обеспечивая считывание каждого аргумента до разделителя.

Одной из проблем многопоточного клиент-серверного приложения является тот факт, что по некоторым операциям клиенты конфликтуют по ресурсам, в связи с чем при одновременном обращении к этим ресурсам на сервере может возникнуть конфликт, приводящий к ошибке, либо как минимум к затиранию информации другого пользователя. В связи с этим в проекте были применены средства синхронизации (именованные семафоры) для таких операций, как добавление нового пользователя, новых сообщений и т.д.

В ходе курса были получены основные навыки разработки прикладных сетевых приложений; навыки разработки собственных прикладных протоколов обмена; навыки разработки многопоточных сетевых приложений; навыки работы с утилитами статического/динамического анализа кода, программами-анализаторами сетевого трафика; знания по организации сетевого обмена по транспортным протоколам TCP и UDP. 

На примере данной разработки были изучены основные приемы использования протокола транспортного уровня TCP – транспортного механизма, предоставляющего поток данных, с предварительной установкой соединения, за счёт этого дающего уверенность в достоверности получаемых данных, осуществляющего повторный запрос данных в случае потери данных и устраняющего дублирование при получении двух копий одного пакета. Данный механизм, в отличие от UDP, гарантирует, что приложение получит данные точно в такой же последовательности, в какой они были отправлены, и без потерь. При подключении нового клиента создается новый сокет, что значительно упрощает создание многоклиентского приложения на основе нитей. Нити в многопоточном приложении позволяют таким образом  "разгрузить"  обработку данных от нескольких клиентов сервером, который каждому клиенту в отдельном потоке назначает свой обработчик.

\section{Листинги программ}

Листинги программ находятся по адресу: 

https://github.com/AleksanderBoldyrev/TCP\_UDP\_MAIL.git

\section{Приложение}
\subsection{Листинг вывода Valgrind-анализатора}
 \lstinputlisting[
	label=code:code_4,
	caption={Вывод после работы утилиты Valgrind с параметром --tool=memcheck --leak-check=full},% для печати символ '_' требует выходной символ '\'
]{code_4.txt}

\end{document}

